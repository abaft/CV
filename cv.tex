\documentclass[9pt,a4paper]{article}
\usepackage[margin=0.5in]{geometry}
\usepackage[utf8]{inputenc}
\usepackage[english]{babel}
\usepackage{array}
\setlength{\parindent}{0pt}
\usepackage{multicol}
\usepackage[hidelinks,bookmarks=false]{hyperref}
\def\sPace{\vspace{0.3cm}}
\begin{document}

\begin{center}

  \textsc{\Huge{Benjamin Chalmers}}
  \vspace{0.3cm}

  \noindent\rule{0.8\textwidth}{0.4pt}
  \vspace{0.3cm}

  \textsc{\large{Curriculum Vitae}}
  \vspace{0.2cm}

\end{center}
\begin{multicols*}{2}
  \section*{Contact Details} 
\begin{center}
  \begin{tabular}{p{2cm}p{6cm}}
    \textbf{Email}&\href{mailto:benj@minchalme.rs}{benj@minchalme.rs}\\
    \textbf{Website}&\href{http://benja.minchalme.rs/}{http://benja.minchalme.rs/}\\
    \textbf{github}&\href{http://github.com/abaft}{abaft}\\
    \textbf{Tel.}&07410 446293\\
    \textbf{Address}&Benjamin Chalmers\\
    &\parbox{5cm}{19 Brettenham Crescent\\Ipswich\\Suffolk\\IP4 2UB}\\
  \end{tabular}
\end{center}

  \section*{Education}
  \vspace*{-.3cm}
  University Of Leeds:
  %\newline\textit{*Provisional Grade}
	\newline \textit{On track for 2:1}
  \vspace*{-6mm}
\begin{center}
  \begin{tabular}{p{2cm} p{0.6cm} p{5.4cm}}
    \textbf{First Year}&\textbf{70}&Computer Architecture\\
	2017-18&\textbf{63}&Fundamental Math Concepts\\
	    &\textbf{96}&Procedural Programming\\
			&\textbf{74}&Computer Processors\\
			&\textbf{65}&Object Oriented Programming\\
			&\textbf{78}&Programming Project
  \end{tabular}
\end{center}
  St Josephs Collage, Ipswich:
  \vspace*{-.6cm}
\begin{center}
  \begin{tabular}{p{2cm} p{0.6cm} p{5.4cm}}
    \textbf{A Levels}&\textbf{A}&Maths\\
	2017&\textbf{A}&Further Maths\\
	    &\textbf{B}&Chemistry\\
	    &\textbf{B}&Physics\\
\\
    \textbf{GCSEs}
        2015&& 10 A*-B including A* Maths, A* English Literature
  \end{tabular}
\end{center}


\section*{Personal Profile}
Enthusiastic and technically minded individual who enjoys a challenge and revels in solving problems.
Hard working with a persistent desire to learn and understand.
Currently in my second year studying toward a BSc in Computer Science at the University of Leeds.

I have, to date, gained experience in computing and electronics through a plethora of personal projects, hobbies, and some work experience.

\section*{Projects}
%\textit{Experience listed here was done `for love' in addition to my academic studies}
\textit{Some of the projects I have done and found interesting enough to comment on here}

\paragraph{Building Processors/FPGA Softcores} In my high school years I built an 8 bit processors in 74 series logic.
At university I discovered VeriLog/VHDL and FPGAs and realised that I could code my own processors without having to deal with PCBs or a ton of wires (or silicon lithography).

\paragraph{radiotutor.uk} My friend approached me to write a web application for his e-learning platform.
I took the opportunity to learn Go, I ended up settling on GoLang's gin as the web framework as it struck a good balance between performance and ease of use.
Worked with databases (MySQL and Redis), and tools such as Docker and nginx for deployment and load balancing. (given our limited scope load balancing turned out to be completely redundant)

\paragraph{High Altitude Balloons} For the sake of pretty areal photographs I decided to try and launch my own meteorological balloon.
I developed a interrupt driven RTTY radio GPS tracker device using a megaAVR microcontroller and a little radio transmitter.
This taught me about assembler and hardware interfaces such as I$^2$C, UART and SPI;
as well as elements computer architecture as enforced by my computer science course.
It tested my programming and construction abilities, to work with a very limited microcontroller in terms of RAM and PROM space in an environment where reliability really matters.

\section*{Work Experience}

\paragraph{BT} I spent a month doing work experience with BTs research and innovation division in martleshem.
I was working on developing a demonstration of what CV2X could provide by building robotic cars communicating P2P and over a network.
I built the cars and wrote some server client software to control them (In C and GoLang) using bluetooth for P2P and a 4G `over the top' connection for communication with the control server which I wrote and hosted in AWS.

This taught me a little about office life and the crazy number of acronyms for everything.

\section*{Achievements}
\begin{itemize}
	\item Full amateur radio licence

	\item D of E Silver and Bronze

  \item Treasurer of CompSoc (School of Computing Society) and Sectary of Rifle Soc (our university rifle club)

	\item Course rep
\end{itemize}


%\iffalse
\section*{Technologies I'm comfortable with}
\begin{multicols*}{2}
\begin{itemize}
\item C
\item Golang
\item Verilog
\item \LaTeX 
\item Git
\item Linux
\item Java
\item PHP, Perl, Lua, tcl, R, python, javascript, m4, any scripting language within a day
\end{itemize}
\end{multicols*}
%\fi
\vfill
\textit{References available upon request}
\end{multicols*}

\end{document}
