\documentclass[9pt,a4paper]{article}
\usepackage[margin=0.5in]{geometry}
\usepackage[utf8]{inputenc}
\usepackage[english]{babel}
\usepackage{array}
\setlength{\parindent}{0pt}
\usepackage{multicol}
\usepackage[hidelinks,bookmarks=false]{hyperref}
\def\sPace{\vspace{0.3cm}}
\begin{document}

\begin{center}

  \textsc{\Huge{Benjamin Chalmers}}
  \vspace{0.3cm}

  \noindent\rule{0.8\textwidth}{0.4pt}
  \vspace{0.3cm}

  \textsc{\large{Curriculum Vitae}}
  \vspace{0.2cm}

\end{center}
\begin{multicols*}{3}
  \section*{Contact Details} 
  \vspace{-0.5cm}
\begin{center}
  \begin{tabular}{p{1.2cm}p{6cm}}
    \textbf{Email}&\href{mailto:benj@minchalme.rs}{benj@minchalme.rs}\\
    \textbf{Website}&\href{http://benja.minchalme.rs/}{benja.minchalme.rs}\\
    \textbf{Github}&\href{http://github.com/abaft}{abaft}\\
    \textbf{Tel.}&07410 446293\\
    \textbf{Address}&Benjamin Chalmers\\
    &\parbox{5cm}{19 Brettenham Crescent\\Ipswich\\Suffolk\\IP4 2UB}\\
  \end{tabular}
\end{center}

  \vspace{-.5cm}
  \section*{Education}
  \vspace*{-.3cm}
  BSc Computer Science 
  \newline (University Of Leeds 2.1)
  %\newline\textit{*Provisional Grade}
  \vspace*{-6mm}
\begin{center}
  \begin{tabular}{p{1.2cm}p{5.4cm}}
    \textbf{Year 1}&Computer Architecture\\
	2017-18&Discrete Maths\\
			&Computer Processors\\
	    &Procedural Programming\\
      &OOP\vspace{.3cm}\\
    \textbf{Year 2}&Software Engineering\\
    2018-19&Algorithms\\
    &Networks\\
    &Compiler Design\\
    &Formal Languages\vspace{.3cm}\\
    \textbf{Year 3}&Graphics\\
    2019-20&Machine Learning\\
    &Secure Computing\\
    &Cryptograpy\\
    &Graph Alogrithms\\
    &Final Project
  \end{tabular}
\end{center}
  St Josephs Collage, Ipswich:
  \vspace*{-.6cm}
\begin{center}
  \begin{tabular}{p{1.5cm} p{0.3cm} p{5.4cm}}
    \textbf{A Levels}&\textbf{A}&Maths\\
G	2017&\textbf{A}&Further Maths\\
	    &\textbf{B}&Chemistry\\
	    &\textbf{B}&Physics\\
  \end{tabular}
  \begin{tabular}{p{1.5cm} p{5.7cm}}
    \textbf{GCSEs}
        & 10 A*-B including:\\
        2015&A* Maths\\
        &A* English Literature
  \end{tabular}
\end{center}

\section*{Personal Profile}
\vspace{-.3cm}
Enthusiastic and technically minded individual who enjoys a challenge and revels in solving problems.
Hard working with a persistent desire to learn and understand.

I have, to date, gained experience in computing and electronics through a plethora of projects, hobbies, and work experience;
in conjunction with my degree program.

\section*{Projects}
\vspace*{-.3cm}
%\textit{Experience listed here was done `for love' in addition to my academic studies}
%\textit{Some of the projects I have done and found interesting enough to comment on here}
\textit{Things I found fun}
\vspace*{-.3cm}

\paragraph{Processors} In my high school years I built an 8 bit processor in 74 series logic.
At university I discovered VeriLog/VHDL and FPGAs and realised that I could code my own logic components without having to deal with enormous numbers of chips.
This was explored further in the processors module in my degree, in which I implemented the `HACK' educational (nand2tetris) architecture on an FPGA as an extension to my coursework.

\paragraph{Radiotutor.uk} My friend approached me to architect and write the software backend for an e-learning platform.
I took the opportunity to familiarise myself with the web services provided by Amazon.
Creating RESTful, and an event stream API;
experimenting with NoSQL and graphQL database solutions.
The software was maintainably written in GoLang, and C++; using git for the VCS.
I also implemented automated testing and deployment.

\paragraph{High Altitude Balloons} For the sake of pretty aerial photographs I decided to try and launch my own meteorological balloon.
I developed an interrupt driven RTTY radio GPS tracker device using a tinyAVR microcontroller and a little radio transmitter.
This taught me about hardware interfaces such as I$^2$C, UART and SPI;
as well as elements of computer architecture (timers, addressing, and register).
It tested my creativity, programming, and construction abilities.
To work with a very limited microcontrollers, and create reliable, robust software.

\section*{Work Experience}

\paragraph{BT - Summer 2018} I spent a month doing work experience with BTs research and innovation division in Martleshem.
I was working on developing a demonstration of what CV2X could provide by building robotic cars communicating P2P and over a network.
I built the cars and wrote some client server software to control them (In C and GoLang) using bluetooth for the P2P communication, and a 4G `over the top' connection for communication with a control server which I wrote.

This taught me about office life, as well as the outstanding number of acronyms for everything and anything.

\paragraph{BT - Summer 2019} I returned to the research and innovation division at BT Martlesham to become a paid intern in the summer of 2019.
During this time worked with a team researching tooling for software engineers at BT.
My work was primarily to find a way for software engineers to better understand legacy code bases.
To do this I created a tool (in C++) which would perform runtime hot code path analysis on programs running in the java verbal machine.
This vastly increased my underlying understanding about how the JVM, and it's garbage collector, works.
In this role I acquired skills querying a graphing database in graph QL (neo4j, using it to store profiling data), providing a powerful new way to think about databases beyond the relation databased taught in my degree.

\section*{Achievements}
\begin{itemize}
	\item Full amateur radio licence

	\item D of E Silver and Bronze

  \item Treasurer of CompSoc (School of Computing Society), Secretary \& Captain of Rifle Soc (our university rifle club), and Social Secretary of the University Caving Club.

	\item Course rep in second year.
\end{itemize}

%\iffalse
\section*{Technologies I'm comfortable with}
\begin{multicols*}{2}
\begin{itemize}
\item C/C++
\item Golang
\item Verilog
\item \LaTeX 
\item Git
\item GNU/Linux
\item Java
\item AWS
\item PHP, Perl, Lua, tcl, R, python, javascript
\end{itemize}
\end{multicols*}
%\fi
\vfill
\textit{References available upon request}
\end{multicols*}

\end{document}
