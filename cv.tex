\documentclass[9pt,a4paper]{article}
\usepackage[margin=0.5in]{geometry}
\usepackage[utf8]{inputenc}
\usepackage[english]{babel}
\usepackage{array}
\setlength{\parindent}{0pt}
\usepackage{multicol}
\usepackage[hidelinks,bookmarks=false]{hyperref}
\def\sPace{\vspace{0.3cm}}
\begin{document}

\begin{center}

  \textsc{\Huge{Benjamin Chalmers}}
  \vspace{0.3cm}

  \noindent\rule{0.8\textwidth}{0.4pt}
  \vspace{0.3cm}

  \textsc{\large{Curriculum Vitae}}
  \vspace{0.2cm}

\end{center}
\begin{multicols*}{2}
  \section*{Contact Details} 
\begin{center}
  \begin{tabular}{p{2cm}p{6cm}}
    \textbf{Email}&\href{mailto:benj@minchalme.rs}{benj@minchalme.rs}\\
    \textbf{Website}&\href{http://benja.minchalme.rs/}{http://benja.minchalme.rs/}\\
    \textbf{github}&\href{http://github.com/abaft}{abaft}\\
    \textbf{Tel.}&07410 446293\\
    \textbf{Address}&Benjamin Chalmers\\
    &\parbox{5cm}{19 Brettenham Crescent\\Ipswich\\Suffolk\\IP4 2UB}\\
  \end{tabular}
\end{center}

  \section*{Education}
  \vspace*{-.3cm}
  University Of Leeds:
  \newline\textit{*Provisional Grade}
  \vspace*{-6mm}
\begin{center}
  \begin{tabular}{p{2cm} p{0.6cm} p{5.4cm}}
    \textbf{First Year}&\textbf{70}*&Computer Architecture\\
	2017-18&\textbf{63}*&Fundamental Math Concepts\\
	    &\textbf{96}*&Procedural Programming\\
	    &\textbf{61}*&Professional Computing\\
	    &\textbf{60}*&Calculus \& Math Analysis\\
  \end{tabular}
\end{center}
  St Josephs Collage, Ipswich:
  \vspace*{-.6cm}
\begin{center}
  \begin{tabular}{p{2cm} p{0.6cm} p{5.4cm}}
    \textbf{A Levels}&\textbf{A}&Maths\\
	2017&\textbf{A}&Further Maths\\
	    &\textbf{B}&Chemistry\\
	    &\textbf{B}&Physics\\
\\
    \textbf{GCSEs}
        2015&& 10 A*-B including A* Maths, A* English Literature
  \end{tabular}
\end{center}


\section*{Personal Profile}
Enthusiastic and technically minded individual who enjoys a challenge and revels in solving problems.
Hard working with a persistent desire to learn and understand.
Currently in first year studying toward a BSc in Computer Science at the University of Leeds.

\section*{Core Skills}
\begin{itemize}
\item Grounding in mathematics providing a robust structure to think about and model problems.

\item Practical problem solving skills and algorithmic thinking abilities.

\item Communication skills and ability to work cohesively within a team.

\item An ability to quickly adapt myself to new environments and learn new skills efficiently.
\end{itemize}

\section*{Projects}
%\textit{Experience listed here was done `for love' in addition to my academic studies}
\textit{Some of the projects I have done and found interesting}

\paragraph{Micro Controllers} In an attempt to track a meteorological balloon going into near space I worked intimately with AVR Micro Controllers.
I developed a interrupt driven RTTY radio GPS tracker device using a megaAVR microcontroller.
This taught me about assembler and hardware interfaces such as I$^2$C, UART and SPI;
as well as elements computer architecture as enforced by my computer science course.
It also stretched my programming abilities, to work with a very limited microcontroller in terms of RAM and PROM space.
\vspace*{-0.1cm}
\paragraph{Radio} I discovered amateur radio a few years ago (attaining the full licence) to experiment with digital data transmission modes and packet radio.
This led to a wider interest in computer networking (TCP/IP and UDP/IP).
I enjoy working with almost all communications protocols and encryption.
\vspace*{-0.1cm}
\paragraph{Web Development} In senior school I was a member of a `Young Enterprise' team.
I was tasked to develop an e-commerce platform to sell our products.
Through this I worked with PHP, mySQL, AJAX, nginx, apache, linux ect
This `Young Enterprise' experience, along with ICAEW's BASE competition (the team I was in came within the top 8 of 500 nation wide) taught me about business and management providing valuable team working skills.
\vspace*{-0.1cm}
\paragraph{Electronics/FPGAs} I've always enjoyed building machines with discrete electronics;
engineering solutions to all sorts of things with various TTL ICs.
More Recently FPGAs have been my primary source of entertainment.
The ability to describe your circuit and flash it onto a general purpose gate array is exceptionally cool.
I recently, as part of a computer processors module, decided to implement the architecture we were discussing (called HACK) in verilog and flash it onto an FPGA.

\iffalse
\section*{Some Language/Software Skills}
\begin{multicols*}{2}
\begin{itemize}
\item C/C++
\item JavaScript
\item HTML5/CSS3
\item PHP
\item SQL
\item Python
\item Golang
\item Verilog
\item \LaTeX 
\item Git
\item Linux
\item BASH scripting
\item MS Office
\end{itemize}
\end{multicols*}
\fi
\vfill
\textit{References available upon request}
\end{multicols*}

\end{document}
